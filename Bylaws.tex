\documentclass{article}
\usepackage[letterpaper, margin = 1 in]{geometry}
\usepackage{enumerate}
\usepackage{paralist}
\usepackage[english]{babel}
\usepackage{csquotes}
\usepackage{textcomp}
\begin{document}
	\pagenumbering{gobble} 
	\title{\textbf {\uppercase{Bylaws}}}
	\author{\uppercase{Tau Beta Pi CA-Alpha, Inc.} \\ \textsc{A California Nonprofit Public Benefit Corporation}}
	\date{November 3, 2016}
	\maketitle
	\clearpage

	\pagenumbering{arabic}

	\section{Associations}
	\subsection{Association with Tau Beta Pi Association, Inc.}
	These Bylaws shall govern the proceedings of this Chapter in all matters not expressly provided for in the Constitution and Bylaws of The Tau Beta Pi Association, Inc.
	\subsection{Association with University of California, Berkeley}
	This chapter shall conform to such rules and regulations of the University of California at Berkeley and of its College of Engineering and College of Chemistry as may apply to student organizations and honor societies.
	\subsection{Addressing of Regulatory Conflicts}
	In the event of conflict between the Constitution and Bylaws of The Tau Beta Pi Association, Inc., and the rules and regulations of the University of California at Berkeley and/or its College of Engineering and/or its College of Chemistry, the rules of the University of California at Berkeley and/or the College of Engineering and/or its College of Chemistry shall prevail, and the Secretary- Treasurer of the Association shall be notified of the circumstances of conflict.
	\subsection{Abbreviations of References}
	Abbreviated references herein to the Constitution and Bylaws of the Association and to the various Bylaws of this document shall be made in the form illustrated by the following examples:
	\begin{enumerate}[\indent a)]
	\item C-VI, 1 = National Constitution, Article VI, Section 1. 
	\item B-V, 5.02 = National Bylaw, Article V, Section 5.02.
	\item ASUC = Associated Students of the University of California 
	\end{enumerate}
	
	\section{Offices}
	\subsection{Principal Office}
	The principal office of the corporation for the transaction of its business is located in Alameda County, California, at 101 O\textquotesingle Brien Hall, University of California, Berkeley, CA 94720.
	\subsection{Change of Address}
	The address of the corporation\textquotesingle s principal office can be changed only by amendment of these bylaws and not otherwise.
	
	\subsection{Other Offices}
	The corporation may also have offices at such other places, within or without the State of California, where it is qualified to do business, as its business may require and as the board of directors may, from time to time, designate.
	
	\section{Purposes}
	\subsection{Objectives and Purposes}
	The primary objectives and purposes of this corporation shall be to recognize, celebrate, and champion excellence and integrity in engineering at the University of California at Berkeley, and to encourage and promote the engineering discipline in the community and beyond. To do so, the Chapter will host professional development events, organize industry opportunities, provide socials and activities for the engineering community, and encourage outreach and public service to promote engineering and give back to the community.
	\subsection{Nonprofit Nature of Purposes}
	This organization is a nonprofit public benefit corporation and is not organized for the private gain of any person. It is organized under the Nonprofit Public Benefit Corporation Law for: public and charitable purposes. The Specific purposes for which this corporation is organized include, but are not limited to: delivery of programs for scientific, educational, and recreational opportunities.
	
	The corporation is organized and operated exclusively for charitable purposes within the meaning of Section 501(c)(3) of the Internal Revenue Code. Notwithstanding any other provision of these Articles, the corporation shall not carry on any other activities not permitted to be carried on 
	\begin{inparaenum}[\itshape 1\upshape)]
		\item by a corporation exempt from federal income tax under Section 501(c)(3) of the Internal Revenue Code, corresponding section of any future federal tax code, or 
		\item by an organization, contributions to which are deductible under section 170(c)(2) of the Internal Revenue Code, or corresponding section of any future federal tax code.
	\end{inparaenum}
	
	No substantial part of the activities of this corporation shall consist of carrying on propaganda or otherwise attempting to influence legislation; nor shall the corporation participate or intervene in any political campaign (including the publishing or distribution of statements) on behalf of any candidate for public office.
	
	\section{Directors}
	\subsection{Number}
	The corporation shall have no less than three (3) directors and collectively they shall be known as the Board of Directors. The corporation shall not have more than eight (8) directors. All directors shall attain the age of eighteen (18) years prior to taking office. The number may be changed by amendment of this bylaw, or by repeal of this bylaw and adoption of a new bylaw, as provided in these bylaws.
	
	\subsection{Powers}
	Subject to the provisions of the California Nonprofit Public Benefit Corporation law and any limitations in the articles of incorporation and bylaws relating to action required or permitted to be taken or approved by the members, if any, of this corporation, the activities and affairs of this corporation shall be conducted and all corporate powers shall be exercised by or under the direction of the board of directors.
	
	\subsection{Duties}
	It shall be the duty of all the directors to:
	\begin{enumerate}[\indent (a)]
		\item Perform any and all duties imposed on them collectively or individually by law, by the articles of incorporation of this corporation, or by these bylaws;
		\item Appoint and remove, employ and discharge, and, except as otherwise provided in these bylaws, prescribe the duties and fix the compensation, if any, of all officers, agents, and employees of the corporation;
		\item Supervise all officers, members, agents, and employees of the corporation to assure that their duties are performed properly;
		\item Meet at such times and places as required by these bylaws;
		\item Register their addresses and specify preferred methods of electronic transmission with the Secretary of the Corporation. Notices of meetings mailed to them at such addresses or sent via director-specified methods of electronic transmission shall be valid notices thereof. %This was changed 110316 at the suggestion of Matt, to make "specified methods of electronic transmission" more clear.
	\end{enumerate}
	
	\subsection{Terms of Office}
	Each director shall hold office for 2 years or until a new set of directors have been selected as specified in these bylaws, and until his or her successor is decided and qualifies.
	\subsection{Compensation}
	Directors shall serve without compensation. They shall be allowed reasonable advancement or reimbursement of expenses incurred in the performance of their regular duties as specified in Section 3 of this Article. Directors may not be compensated for rendering services to the corporation in any capacity other than director unless such other compensation is reasonable and is allowable under the provisions of Section 6 of this Article. Any payments to directors shall be approved in advance in accordance with this corporation\textquotesingle s conflict of interest policy, as set forth in Article 13 of these bylaws.
	\subsection{Restriction Regarding Interested Directors}
	Notwithstanding any other provision of these bylaws, not more than forty-nine percent (49\%) of the persons serving on the board may be interested persons. For purposes of this section, ``interested persons'' means either: 
	\begin{enumerate}[\indent (a)]
		\item Any person currently being compensated by the corporation for services rendered it within the previous twelve (12) months, whether as a full- or part-time officer or other employee, independent contractor, or otherwise, excluding any reasonable compensation paid to a director as director; or
		\item Any relative within one degree or two degrees of consanguinity.
	\end{enumerate}
	\subsection{Place of Meetings}
	Meetings shall be held at the principal office of the corporation unless otherwise provided by the board or at such place within or without the State of California which has been designated from time to time by resolution of the board of directors. In the absence of such designation, any meeting not held at the principal office of the corporation shall be valid only if held on the written consent of a quorum of directors given before the meeting and filed with the secretary of the corporation and after all board members have been given written notice of the meeting as hereinafter provided for special meetings of the board.
	
	Any meeting, regular or special, may be held by conference telephone, electronic video screen communication, or other communications equipment. Directors must be able to effectively communicate and participate in the meeting. Specifically, participation through electronic means constitutes presence in person if all of the following apply:
	\begin{enumerate}[\indent (a)]
		\item Each director participating in the meeting can communicate with all of the other directors concurrently;
		\item	Each director is provided the means of participating in all matters before the board, including, without limitation, the capacity to propose, or to interpose, an objection to a specific action to be taken by the corporation; and
		\item	The corporation adopts and implements some means of verifying:
		\begin{enumerate}[\indent 1)]
			\item that all persons participating in the meeting are directors of the corporation or are otherwise entitled to participate in the meeting, and
			\item that all actions of, or votes by the board are taken and cast only by directors and not by persons who are not directors. 
		\end{enumerate}	
	\end{enumerate}
	
	
	\subsection{Regular and Annual Meetings}
	Annual meetings of directors shall be held between the dates of March 1 and April 30, unless such day falls on a legal holiday, in which event the regular meeting shall be held at the same hour and place on the next business day. 
	
	At the annual meeting of directors, directors shall be elected by the board of directors in accordance with this section. Cumulative voting by directors for the election of directors shall not be permitted. The candidates receiving the highest number of votes up to the number of directors to be elected shall be elected. Each director shall cast one vote, with voting being by ballot only. 
	
	\subsection{Special Meetings}
	Special meetings of the board of directors may be called by the chairperson of the board, the president, the vice president, the secretary or by any two directors, and such meetings shall be held at the place, within or without the State of California, designated by the person or persons calling the meeting, and in the absence of such designation, at the principal office of the corporation.
	
	\subsection{Notice of Meetings}
	Regular meetings of the board may be held without notice.
	
	Special meetings of the board shall be held upon ten (10) days notice by first class mail or five (5) days notice given personally or by telephone or other electronic means of communication. If sent by mail or other electronic means of communication, the notice shall be deemed to be given on its deposit in the mail or on execution of the ``send'' command. Such notices shall be addressed to each director at his or her address as shown on the books of the corporation.
	
	% [I'm not sure why this section is here and don't know if it's necessary. Please help. - ZL] Notice of the time and place of holding an adjourned meeting need not be given to absent directors if the time and place of the adjourned meeting is fixed at the meeting adjourned and if such adjourned meeting is held no more than twenty-four (24) hours from the time of the original meeting. Notice shall be given of any adjourned regular or special meeting to directors absent from the original meeting if the adjourned meeting is held more than twenty-four (24) hours from the time of the original meeting.  %
	\subsection{Contents of Notice}
	Notice of meetings not herein dispensed with shall specify the place, day, and hour of the meeting. The purpose of any board meeting need not be specified in the notice.
	\subsection{Waiver of Notice and Consent to Holding Meetings}
	The transactions of any meeting of the board, however called and noticed or wherever held, are as valid as though the meeting had been duly held after proper call and notice, provided a quorum, as hereinafter defined, is present and provided that before the meeting each director not present signs a waiver of notice or a consent to holding the meeting. All such waivers or consents shall be filed with the corporate records or made a part of the minutes of the meeting.
	\subsection{Quorum for Meetings}
	A quorum shall consist of a majority of the directors.
	
	Except as otherwise provided in these bylaws or in the articles of incorporation of this corporation, or by law, no business shall be considered by the board at any meeting at which a quorum, as hereinafter defined, is not present, and the only motion which the chair shall entertain at such meeting is a motion to adjourn.
	
	When a meeting is adjourned for lack of a quorum, it shall not be necessary to give any notice of the time and place of the adjourned meeting or of the business to be transacted at such meeting, other than by announcement at the meeting at which the adjournment is taken. %, except as provided in Section 10 of this Article. Removed at Matt's suggestion.
	
	The directors present at a duly called and held meeting at which a quorum is initially present may continue to do business notwithstanding the loss of a quorum at the meeting due to a withdrawal of directors from the meeting, provided that any action thereafter taken must be approved by at least a majority of the required quorum for such meeting or such greater percentage as may be required by law, or the articles of incorporation or bylaws of this corporation.
	
	\subsection{Majority Action as Board Action}
	Every act or decision done or made by a majority of the directors present at a meeting duly held at which a quorum is present is the act of the board of directors, unless the articles of incorporation or bylaws of this corporation, or provisions of the California Nonprofit Public Benefit Corporation Law, particularly those provisions relating to appointment of committees (Section 5212), approval of contracts or transactions in which a director has a material financial interest (Section 5233), and indemnification of directors (Section 5238e), require a greater percentage or different voting rules for approval of a matter by the board.
	
	\subsection{Conduct of Meetings}
	Meetings of the board of directors shall be presided over by the chairperson of the board, or, if no such person has been so designated or, in his or her absence, the president of the corporation or, in his or her absence, by the vice president of the corporation or, in the absence of each of these persons, by a chairperson chosen by a majority of the directors present at the meeting. The secretary of the corporation shall act as secretary of all meetings of the board, provided that, in his or her absence, the presiding officer shall appoint another person to act as secretary of the meeting.
	
	Meetings shall be governed by common sense, courtesy and appropriate and respectful decorum, and may not be inconsistent with or in conflict with these bylaws, With the articles of incorporation of this corporation, or with provisions of law. A majority of the directors present at a meeting may adjourn from time to time until the time fixed for the next regular meeting of the board.
	
	\subsection{Action by Unanimous Written Consent Without Meeting}
	Any action required or permitted to be taken by the board of directors under any provision of law may be taken without a meeting, if all members of the board shall individually or collectively consent in writing to such action. For the purposes of this Section only, ``all members of the board'' shall not include any ``interested director'' as defined in Section 5233 of the California Nonprofit Public Benefit Corporation Law. Such written consent or consents shall be filed with the minutes of the proceedings of the board. Such action by written consent shall have the same force and effect as the unanimous vote of the directors. Any certificate or other document filed under any provision of law which relates to action so taken shall state that the action was taken by unanimous written consent of the board of directors without a meeting and that the bylaws of this corporation authorize the directors to so act, and such statement shall be prima facie evidence of such authority.
	\subsection{Vacancies}
	Vacancies on the board of directors shall exist 
	\begin{inparaenum}[\itshape 1\upshape)]
		\item on the death, resignation, or removal of any director; and
		\item whenever the number of authorized directors is increased.
	\end{inparaenum}
	
	The board of directors may declare vacant the office of a director who has been declared of unsound mind by a final order of court, or convicted of a felony, or been found by a final order or judgment of any court to have breached any duty under Section 5230 and following of the California Nonprofit Public Benefit Corporation Law.
	
	If this corporation has any members, then, if the corporation has fewer than fifty (5O) members, directors may be removed without cause by a majority of all members, or, if the corporation has fifty (5O) or more members, by vote of a majority of the votes represented at a membership meeting at which a quorum is present.
	
	If this corporation has no members, directors may be removed without cause by a majority of the directors then in office.
	
	Any director may resign effective upon giving written notice to the chairperson of the board, the president, the secretary, or the board of directors, unless the notice specifies a later time for the effectiveness of such resignation. No director may resign if the corporation would then be left without a duly elected director or directors in charge of its affairs, except upon notice to the Attorney General.
	
	Vacancies on the board may be filled by approval of the board or, if the number of directors then in office is less than a quorum, by
	\begin{inparaenum}[\itshape 1\upshape)]
		\item the unanimous written consent of the directors then in office; 
		\item the affirmative vote of a majority of the directors then in office at a meeting held pursuant to notice or waivers of notice complying with this Article of these bylaws; or 
		\item a sole remaining director. 
	\end{inparaenum}
	If this corporation has members, however, vacancies created by the removal of a director may be filled only by the approval of the members. The members, if any, of this corporation may elect a director at any time to fill any vacancy not filled by the directors.
	
	A person elected to fill a vacancy as provided by this Section shall hold office until the next annual election of the board of directors or until his or her death, resignation, or removal from office.
	
	\subsection{Non-liability of Directors}
	The directors shall not be personally liable for the debts, liabilities, or other obligations of the corporation, unless such obligations are the result of an \textit{ultra vires} act by a director, in which case the director who commited the act will be responsible.
	% We need a clause here for liability of directors for unauthorized action 
	% Clause added 9/21. But, it seems standard for bylaws to omit ultra vires acts
	
	\subsection{Indemnification by Corporation of Directors, Officers, Members, Employees, and Other Agents}
	To the extent that a person who is, or was, a director, officer, member, employee, or other agent of this corporation has been successful, on the merits or otherwise, in defense of any civil, criminal, administrative, or investigative proceeding brought to procure a judgment against such person by reason of the fact that he or she is, or was, an agent of the corporation, or has been successful in defense of any claim, issue, or matter, therein, such person shall be indemnified against expenses actually and reasonably incurred by the person in connection with such proceeding. %"on merit or otherwise" added on Matt's suggestion
	
	If such person either settles any such claim or sustains a judgment against him or her, then indemnification against expenses, judgments, fines, settlements, and other amounts reasonably incurred in connection with such proceedings shall be provided by this corporation but only to the extent allowed by, and in accordance with the requirements of, Section 5238 of the California Nonprofit Public Benefit Corporation Law.
	
	\subsection{Insurance for Corporate Agents}
	The board of directors may adopt a resolution authorizing the purchase and maintenance of insurance on behalf of any agent of the corporation (including a director, officer, employee, or other agent of the corporation) against any liability other than for violating provisions of law relating to self-dealing (Section 5233 of the California Nonprofit Public Benefit Corporation Law) asserted against or incurred by the agent in such capacity or arising out of the agent\textquotesingle s status as such whether or not the corporation would have the power to indemnify the agent against such liability under the provisions of Section 5238 of the California Nonprofit Public Benefit Corporation Law.
	
	\section{Officers and Members}
	\subsection{Eligibility of Members}
	The Chapter shall be composed of members chosen from among eligible students and alumni on the bases of distinguished scholarship or professional attainment and exemplary character, as enunciated in C-VIII and in The Eligibility Code of the Association. Only currently registered students, faculty, and staff may be active members in a registered student organization. Only active members may vote or hold office. (ASUC)
	\subsection{Active and Inactive Members}
	The Chapter shall have full control over its individual affairs, subject to the Constitution and Bylaws of the Association, which control shall be exercised by the active membership of the Chapter and its Advisory Board.
	
	Membership of this Chapter shall be as defined in C-VI, 1, and only active undergraduate and graduate-student members, as defined therein and within Section 13 of this Article, shall have the privilege of voting on new members and on the amounts of the initiation fee, dues, fines, and assessments. %Changed to not conflict with Section 13. It looks like MI-G got away with not referencing the national bylaws whatsoever.
	
	A student may become inactive only under the provisions of B-VI, 6.05. A student may be declared ``temporarily inactive'' for the purposes of conducting official business by a four-sevenths vote of the Advisory Board and a majority of all elected officers (both present and non-present). Such a ``temporarily inactive'' student is reactivated by a four-sevenths vote of the Advisory Board and a majority of all elected officers (both present and non- present) when the official business is completed.
	\subsection{Number of and Initial Officers}
	The officers of this Chapter shall be a President, first Vice President, second Vice President, Recording Secretary, Corresponding Secretary, and a Treasurer, who shall all be active members of the Chapter, and who shall make up the Executive Committee of the Chapter; and four Alumnus advisors, preferably elected from among the faculty of, or the graduate students enrolled in, the College of Engineering. Officers may vote, through procedures described in Section 7 of these bylaws, to elect only one Vice President. Additional officer positions and their duties shall be established as needed by the student officers listed above.
	\subsection{Advisory Board}
	The members of the advisory board of the Chapter shall be those defined in C-VI, 7. The Chapter President shall act as Chairman of the Advisory Board. A quorum of the Advisory Board is defined in Article 7, Section 9 of these bylaws.
	
	In cases where the Chapter has more than four advisors, the Chapter shall, through an election process as described in Article 9, determine which of the advisors will serve on the Advisory Board.
	\subsection{Term Lengths of Advisory Board}
	Alumnus members of the Advisory Board shall serve four-year terms, as specified in C-VI, 7. The Treasurer and Corresponding Secretary shall serve one year terms. All other officers shall serve for one semester or until their successors are duly elected and installed.
	\subsection{Duties of Officers}
	The duties of the officers shall be those usually performed by persons holding such offices, those prescribed by the Constitution and Bylaws of the Association, and those prescribed in these Bylaws or by Chapter action.
	\subsection{Qualification, Election, and Term of Office}
	Any Member may serve as an officer of this corporation. Officers shall be elected by a quorum, as defined by Article 7, Section 9 of these bylaws, during the Election of Officers Meeting, which shall be held within two weeks of the end of the fall and spring semesters. Unless otherwise described in these Bylaws, each officer shall hold office until the next Election of Officers Meeting, or he or she resigns, is removed, or is otherwise disqualified to serve, or until his or her successor shall be elected and qualified, whichever occurs first.
	\subsection{Subordinate Officers}
	The board of directors may appoint such other officers or agents as it may deem desirable, and such officers shall serve such terms, have such authority, and perform such duties as may be prescribed from time to time by the board of directors.
	\subsection{Removal and Resignation}
	Any officer may be removed using the following process:
	\begin{enumerate}[\indent (a)]
		\item \textbf{Indictment:} The Executive Committee shall have the power to draft indictments and approve them by a vote of at least two thirds (2/3) of the size of the quorum. For the purposes of indicting an officer, quorum shall consist of the entire Executive Committee. If the officer being considered for indictment is a member of the Executive Committee, he or she shall be excluded from the Executive Committee for the indictment, trial, and appeal proceedings. An indictment shall list the reason(s) the officer should be removed, including specific instance(s) of misconduct.
		\item \textbf{Trial:} The Executive Committee shall appoint a Prosecutor. The indicted officer shall have the opportunity to defend himself or herself. The President shall preside at the trial as an impartial Judge. If the President is being indicted, or is unable to serve as Judge, then the first Vice President shall serve as Judge instead. If the indicted officer accepts neither the President nor the Vice President as Judge, then the indicted officer must select an Advisor to serve as Judge subject to approval by the Executive Committee. If the indicted officer does not make a selection within seven calendar days of being notified of the indictment, the Executive Committee shall select an Advisor other than the Chief Advisor to serve as Judge. The Judge shall set fair trial procedures, including, but not limited to, setting and adjusting time limits. The indicted officer shall stand trial at the next regularly-scheduled officer meeting that is at least seven calendar days after a Judge and a Prosecutor have been selected. If the indicted officer is unable to attend the scheduled trial, he or she must notify the Judge at least 48 hours before the scheduled trial, and the Judge must provide three alternative times for the trial from which the indicted officer must select one. If the indicted officer is not present at the trial, he or she shall be tried in absentia. The Prosecutor and the indicted officer shall be allowed to present evidence, call witnesses, cross-examine witnesses, and challenge evidence during the allotted time. Upon the expiration of the allotted time (as determined by the Judge), the indicted officer and the Prosecutor must leave the room. The officer corps shall then debate the case and vote. The Judge, the Prosecutor, and the indicted officer are ineligible to vote. Quorum for the trial shall consist of three quarters (3/4) of the current officers who are eligible to vote. An affirmative vote of at least two thirds (2/3) of the eligible officers present shall be necessary to convict. 
		\item \textbf{Appeal:} If convicted, the officer shall have seven calendar days to submit a written appeal to the chapter\textquotesingle s Chief Advisor. The Chief Advisor shall notify the Prosecutor of the appeal and provide a copy thereof. The Prosecutor shall then have seven calendar days to submit a written response to the Chief Advisor. If the Chief Advisor denies the appeal, the convicted officer is removed and his or her position is vacated. A successful appeal can be overturned within fourteen calendar days by an affirmative vote of at least four fifths (4/5) of the total number of officers, excluding the Judge, the Prosecutor, and the convicted officer. If the appeal is overturned, the convicted officer is removed and his or her position is vacated. If the officer corps does not overturn the appeal within fourteen calendar days, the successful appeal stands. 
		\item \textbf{Double jeopardy:} An officer may not be indicted or tried more than once for the same instance(s) of misconduct.
	\end{enumerate}
	\subsection{Vacancies}
	Any vacancy caused by the death, resignation, removal, disqualification, or otherwise, of any officer shall be filled by special elections held by the Executive Committee in accordance with Article 9. In the event of a vacancy in any office other than that of president, such vacancy may be filled temporarily by appointment by the president until such time as the Executive Committee shall fill the vacancy. Vacancies occurring in offices of officers appointed at the discretion of the board may or may not be filled as the board shall determine.
	\subsection{Representative to National Convention}
	The President shall be this Chapter\textquotesingle s delegate to the Convention of the Association. The other officers shall be alternate delegates, in the order given in Section 3 of this Article. The order shall extend into additional officers in the order that those officers were chosen.
	\subsection{Additional Executive Responsibilities}
	The following extra duties shall be required of the officers:
	\begin{enumerate}[\indent (a)]
		\item The President shall see that each officer and/or project leader is provided with a written list of specific duties for which each is responsible. A copy of each list shall be placed in the Recording Secretary\textquotesingle s notebook and in the President\textquotesingle s Book.
		\item The Recording Secretary shall see that each candidate accepting election receives copies of The Constitution and Bylaws and Eligibility Code of The Tau Beta Pi Association, information about Tau Beta Pi, this chapter\textquotesingle s Bylaws, and such other materials as the Chapter may deem desirable.
		\item The Corresponding Secretary shall serve as Secretary of the Advisory Board.
		\item The President and Vice President(s) shall pay a courtesy call on representatives of the Dean of Engineering early in the fall term to discuss ways in which the Chapter may be of service to the College. The substance of such discussions shall be reported to the Chapter at the next meeting. %Changed from "meeting with the Dean."
		\item The President shall cooperate with the Directors of Tau Beta Pi District 15 and shall encourage the Chapter members to participate in the District\textquotesingle s activities.
		\item The Vice President(s) must inform the candidates before the initiation ceremony that the initiation ritual and motto must be kept secret.
	\end{enumerate}
	\subsection{Definition of Active Membership}
	For student members, active membership for a semester shall be defined through the attendance of two of the three following meetings of the chapter:
	\begin{enumerate}[\indent (a)]
		\item Election of New Members
		\item Final Candidate Review
		\item Election of Officers
	\end{enumerate}
	
	For alumnus members, active membership for a semester shall be defined through the attendance of least one of the three above meetings of the chapter.
	
	Until all three of the relevant chapter meetings for active membership determination are held, active members in the prior semester may maintain active membership status through the attendance of at least one chapter event or meeting. Members who initiated in the current semester or immediate previous semester may obtain temporary active status by attending at least one chapter event or meeting. Inactive members may obtain temporary active status by attending at least five chapter events or meetings.
	
	\subsection{Definition of Quorum}
	For the purposes of determining the number of active members needed for quorum in a semester, the number of enrolled members who obtained active membership status by the conclusion of the previous semester is used. %Changed from "active in the prior semester"
	\subsection{Compensation}
	Officers shall serve without compensation. They shall be allowed reasonable advancement or reimbursement of expenses incurred in the performance of their regular duties as specified in Section 6 of this Article. Officers may not be compensated for rendering services to the corporation in any capacity other than director unless such other compensation is reasonable and is allowable under the provisions of Section 6 of this Article. Any payments to officers shall be approved in advance in accordance with this corporation\textquotesingle s conflict of interest policy, as set forth in Article 13 of these bylaws.
	
	\section{Committees}
	\subsection{Executive Committee}
	The Executive Committee, as described in Article 5, Section 3, is delegated any of the powers and authority of the board in the management of the business and affairs of the corporation, except with respect to:
	\begin{enumerate}[\indent (a)]
		\item The approval of any action which, under law or the provisions of these bylaws, requires the approval of the members or of a majority of all of the members.
		\item The filling of vacancies on the board or on any committee that has the authority of the board. 
		\item The fixing of compensation of the directors for serving on the board or on any committee.
		\item The amendment or repeal of bylaws or the adoption of new bylaws.
		\item The amendment or repeal or any resolution of the board which by its express terms is not so amendable or repealable.
		\item The appointment of committees of the board or the members thereof.
		\item The expenditure of corporate funds to support a nominee for director after there are more people nominated for director than can be elected.
		\item The approval of any transaction to which this corporation is a party and in which one or more of the directors has a material financial interest, except as expressly provided in Section 5233(d)(3) of the California Nonprofit Public Benefit Corporation Law.
	\end{enumerate}
	By a majority vote of its members then in office, the board may at any time revoke or modify any or all of the authority so delegated, increase or decrease but not below two (2) the number of its members/ and fill vacancies therein from the members of the board.
	The committee shall keep regular minutes of its proceedings/ cause them to be filed with the corporate records, and report the same to the board from time to time as the board may require.
	
	\subsection{Other Committees}
	The corporation shall have such other committees as may from time to time be designated by officer elections, as detailed in Article 9. Such other committees may consist of persons who are not also members of the board. 
	\subsection{Meetings and Action of Committees}
	Meetings and action of committees shall be governed by, noticed, held, and taken in accordance with the provisions of these bylaws concerning meetings of the board of directors, with such changes in the context of such bylaw provisions as are necessary to substitute the committee and its members for the board of directors and its members, except that the time for regular meetings of committees may be fixed by resolution of the board of directors or by the committee. The time for special meetings of committees may also be fixed by the board of directors. The board of directors may also adopt rules and regulations pertaining to the conduct of meetings of committees to the extent that such rules and regulations are not inconsistent with the provisions of these bylaws.
	
	\section{Chapter Meetings}
	\subsection{Meetings for Election of Officers}
	A meeting shall be called in the Fall and Spring semesters for the purpose of electing officers.
	
	\subsection{Required Meetings}
	There shall be a minimum of eight other required meetings during the year: two introductory meetings, two for the election of candidates, two for final candidate review, and two for electee initiation; there shall be one of each aforementioned type of meeting in both the fall and spring semesters.
	
	\subsection{Special Meetings}
	Special meetings, taking place no sooner than 72 hours before notification, may be called at any time by the President, any alumnus member of the Advisory Board, or upon written request to the President signed by 20\% of the active members of the Chapter.
	
	Notification must arrive 72 hours in advance by mail, e-mail or phone using the most recent contact information submitted to the chapter.
	
	\subsection{Guide for Meetings}
	``Robert's Rules of Order'' shall be the parliamentary guide of the Chapter in all matters not covered in the Constitution and Bylaws of the Association or in these Bylaws.
	
	\subsection{Frequency of Meetings}	
	The Chapter officers (and advisors) shall meet once a week starting within three weeks of the beginning of the fall semester, and continuing throughout the year, with the exception of university or national holidays, in order to organize Chapter activities.
	
	\subsection{Notice of Meetings}
	Regular meetings may be held without notice.
	
	Election Meetings and Final Candidate Review Meetings must be held with a notice of at minimum 14 days by first class mail, delivered personally, or by telephone or other electronic means of communication. If sent by mail or other electronic means of communication, the notice shall be deemed to be given on its deposit in the mail or on execution of the ``send'' command.
	
	Notices sent to the active members and alumnus advisors announcing non-regular meetings shall clearly state the time, place, and purpose of the meeting.
	
	An assembly of active members may object to the time, place, and purpose of a meeting upon written request to the president signed by 30\% of active members of the Chapter. Such written requests must arrive 48 hours in advance by mail, e-mail or phone using the most recent contact information provided by the chapter.
	
	\subsection{Length of Meetings}
	Business meetings, other than the Election Meeting and Final Candidate Review Meeting, shall last no longer than two hours, unless extended by an affirmative vote of three-fourths of the active members present.
	
	\subsection{Absence of Meetings}
	Attendance of officers shall be required at all regular, scheduled officer meetings, unless excused by the President for reasons which he or she judges to be good and sufficient. A member who must miss a regular, scheduled meeting shall submit his or her reasons to the President, in writing, before the meeting, unless the excuse is illness, in which case a notice after the meeting shall be sufficient.
	
	\subsection{Definition of Quorum}
	A quorum for the consideration of routine business shall be a majority of the duly elected officers of the Chapter. For the election of new members, for changing the initiation fee or Chapter dues, for passing an assessment on the members of the Chapter and for approval or disapproval of a proposed amendment to the Constitution of the Association, a quorum shall be three-fourths of the active membership. A quorum for an Advisory Board meeting shall be four-seventh of the members of the Board.
	
	\subsection{Advisory Board Meetings}
	Advisory Board Meetings must be held at least once a year to discuss general health of the chapter and approve or disapprove proposed bylaw amendments in accordance with C-VI,7(b). Meetings shall be held on UC Berkeley campus unless otherwise provided by the board.
	
	Any meeting may be held by conference telephone, electronic video screen communication, or other communications equipment. Advisory board members must be able to effectively communicate and participate in the meeting. Specifically, participation through electronic means constitutes presence in person if all of the following apply:
	\begin{enumerate}[\indent (a)]
		\item Each advisory board member participating in the meeting can communicate with all of the other directors concurrently;
		\item	Each advisory board member is provided the means of participating in all matters before the board, including, without limitation, the capacity to propose, or to interpose, an objection to a specific action to be taken by the chapter.
	\end{enumerate}
	%"advisory board" specified per Matt's suggesting
	
	\subsection{Remote Participation in Meetings}
	The Chapter shall provide mechanisms for which members may participate in the aforementioned chapter meetings remotely, through instant electronic communication or telephone. Members who attend meetings remotely will count towards achieving the quorum required for the given meeting.

	\section{Election and Initiation of New Members}
	\subsection{Time of Member Elections}
	Election of new members shall be held in the Fall and Spring semesters as soon as possible after the eligibility list becomes available.
	\subsection{Provisions for Member Elections}
	All provisions of C-VIII and B-VI shall be strictly followed.
	\subsection{Eligibility of Members}
	Scholastically eligible students in the College of Engineering and in the College of Chemistry are qualified for membership in the Chapter.
	\begin{enumerate}[\indent (a)]
		\item Eligible undergraduate students shall be those of any of the approved programs in the College of Engineering and the Chemical Engineering major of the College of Chemistry. Approved programs within the College of Engineering include Bioengineering, Civil and Environmental Engineering, Electrical Engineering and Computer Science, Computational Engineering Science, Environmental Engineering Science, Engineering Physics, Industrial Engineering and Operations Research, Manufacturing Engineering, Materials Science and Engineering, Mechanical Engineering, Nuclear Engineering, Energy Engineering, as well as the Engineering Undeclared program. Engineering Mathematics and Statistics students are not eligible for membership in Tau Beta Pi.
		\item Eligible graduate students shall be those registered within the College of Engineering or part of the Chemical and Biomolecular Engineering Department.
		\item Members will be chosen and accepted without discrimination on the basis of race, religion, color, national origin, sex, sexual preference, age, or handicap. We will not haze according to California State Law. We will not restrict membership based upon race, color, national origin, religion, sex, gender identity, pregnancy (including pregnancy, childbirth, and medical conditions related to pregnancy or childbirth), physical or mental disability, medical condition (cancer related or genetic characteristics), ancestry, marital status, age, sexual orientation, citizenship, or service in the uniformed services (including membership, application for membership, performance of service, application for service, or obligation for service in the uniformed services.)
		\item As stated in C-VIII, 2, students may be considered eligible for admission into Tau Beta Pi if they are Juniors or Seniors as determined by the requirements of the College of Engineering at the particular school. Class standing at the University of California, Berkeley (UC Berkeley) is defined as follows (per Title IX, Section 800 of the General Regulations of the UC Berkeley Academic Senate): a Junior is a person who has completed 60 or more units toward a degree program, and a Senior is a person who has completed 90 or more units toward a degree program. The UC Berkeley College of Engineering and the College of Chemistry also accept and employ this rule in determining class standing. This judgment is reasoned as follows:
		\begin{enumerate}[\indent 1)]
			\item The engineering curricula in UC Berkeley\textquotesingle s College of Engineering are fluid and flexible. Classes are not required to be taken in a particular order and no "junior only" or "senior only" classes exist.
			\item Many engineers, especially those seeking admission to Tau Beta Pi, advance quickly into upper-division coursework before their junior year due to AP credits and/or community college classes. These students have already reached a high level of academic performance equivalent to junior or senior work as underclassmen. For the CA-A chapter, class standing shall be determined in the same manner, consistent with university regulations.
		\end{enumerate}
	\end{enumerate}
	\subsection{Eligibility of Transfer Students}
	Transfer students shall be eligible for membership per C-VIII, 2 (k).
	\subsection{Discretion of Eligibility}
	The cases of students whose scholastic eligibility is in doubt because of irregularities in standing or curriculum shall be determined by the Advisory Board of the Chapter.
	\subsection{Flexibility of Scholastic Requirements}
	The Chapter may elect to specify scholastic requirements higher than those stated in C-VIII.
	\subsection{Consultation with Faculty}
	The President shall periodically consult with the senior faculty member of the Advisory Board to determine the names of graduate students, alumni, faculty members, and others who may be eligible for membership under the provisions of C-VIII, 3-6. If no member of the Advisory Board is a faculty member, then the President shall consult with an appropriate faculty member.
	\subsection{Exemplary Character Evaluation}
	\begin{enumerate}[\indent (a)]
		\item The Vice Presidents shall establish a set of requirements and deadlines for the purposes of character evaluation. These requirements are subject to a two-thirds approval from the Chapter members.
		\item Scholastically eligible candidates shall be contacted through inviting them to attend an introductory meeting at which the Chapter officers shall explain the purpose of Tau Beta Pi, requirements for membership, and answer any questions. At this time, candidates that wish to be considered for election shall be given a questionnaire to complete by a specified date.
		\item After the specified deadline date the Chapter membership shall review a list of all eligible candidates and formulate a recommendation on each candidate.
		\item At the election meeting the Chapter membership shall vote on the above recommendations with the understanding that the election of candidates can be revoked at the Final Candidate Review Meeting. A three-fourths vote of active members present is required to elect a candidate.
	\end{enumerate}
	
	\subsection{Voting Procedure}
	\begin{enumerate}[\indent (a)]
		\item At the Election Meeting, the Eligibility Code shall first be read.
		\item The names of eligible candidates will be submitted to the Chapter membership.
		\item If any active member has a question about any of the candidates, that candidate\textquotesingle s name shall be removed from the group and considered separately.
		\item A Chapter vote shall be taken on the remainder of the group.
		\item Each candidate removed from the group and/or candidates who did not receive a favorable recommendation from the membership committee shall be considered separately after all groups have been voted on. Each name shall be presented for discussion and voted on before the following name is considered.
		\item After all names have been considered and voted on, there shall be a second individual ballot for each candidate who failed election on the first ballot.
		\item No candidate who fails election on the second ballot shall be considered again unless 25\% of the members present so request it.
		\item No candidate who fails election on the third ballot shall be considered again at this election. He or she may be considered again at the next election if he or she is then eligible.
	\end{enumerate}
	\subsection{Confidence of Elections}
	All members shall keep the election results in absolute confidence so that no candidate shall learn of his or her election except by means of official notification; likewise, no candidate shall be informed of the details of the vote, especially concerning the personal matters discussed at the time of voting.
	
	\subsection{Explanation of Elections Process}
	At the first meeting of electees, the President shall explain the requirements, objectives, and activities of the Association and of the Chapter. Those electees desiring to accept election shall formally state their acceptance by returning the appropriate forms to the President.
	
	\subsection{Financial Exemptions}
	Each electee who refuses initiation for financial reasons shall be interviewed by the Advisory Board, as required by C-VIII, 10 (a).
	
	\subsection{Fulfillment of Requirements}
	Each electee shall be required to complete the requirements set forth by the Vice Presidents in Article 8, Section 8a of these bylaws. Satisfactory completion of the requirements will be determined at the Final Candidate Review Meeting.
	
	\subsection{Final Candidate Review}
	A Final Candidate Review Meeting shall be called at the end of each Fall and Spring semester to evaluate each electee. At this meeting, an individual\textquotesingle s status as an electee may be revoked by a three-fourths affirmative vote under the provisions of B-VI, 6.04 if it is found that the individual in question has failed to satisfactorily complete the requirements specified at the beginning of the term or has shown his or her character to be deficient in any other manner.
	
	\section {Election of Officers}
	\subsection{Timeline of Officer Elections}
	The President, Vice Presidents, Recording Secretary, and any additional officers shall be elected in the fall and spring semesters. The Corresponding Secretary, Treasurer, and an alumnus member of the Advisory Board shall be elected in the spring.
	
	\subsection{Nominations}
	Nominations for officers shall be made from the floor at the Election of Officers Meeting.
	
	\subsection{Transfer of Responsibilities}
	The Election of Officers Meeting shall be held within two weeks after initiation in both the fall and spring semesters. The Chapter Officer Installation Procedure given in the President\textquotesingle s Book shall be employed to install the new officers. During the period following the election until the installation, each officer- elect shall work closely with his or her officer counterpart to learn the duties and responsibilities of the office. Transfer of financial records between the past and newly elected Treasurer is contingent upon the completion of the outgoing Treasurer\textquotesingle s financial report as required by CAB-VI, 6.
	
	\subsection{Definition of Quorum}
	The number of active members present at an Election of Officers Meeting shall constitute a quorum for the meeting, and a majority of the quorum shall be required for election. A vote by the members present will cause the nominee receiving the least number of ballots to be dropped from further consideration until one receives a majority vote.
	
	\subsection{Vacancies}
	If any office becomes vacant between regular elections, a special election shall be held at the next Chapter meeting to fill the vacancy. The officer elected shall serve until the next regular election.
	
	\subsection{Resignations}
	Resignations must be submitted to the Chapter President in writing. The resignation shall become effective 7 days from the date of submission unless the resignation is rescinded. The position of the resigning officer shall then become vacant.
	
	\section{Execution of Instruments, Deposits, and Funds}
	\subsection{Changes to Expenses}
	The expenses of the Chapter shall be borne by initiation fees, project revenues, and by such dues and pro-rata assessments as may be voted on by the Chapter. A three-fourths vote of the active membership shall be required to change any fees or dues, or to levy any assessment. Within one week, the Corresponding Secretary shall inform the Secretary-Treasurer of the Association of any changes in amounts of the Chapter\textquotesingle s initiation fee, dues, or assessments.
	\subsection{Initiation Fees}
	The initiation fee for all initiates shall be payable in advance of initiation, to the Chapter Treasurer. This amount covers the national initiation fee, the National Convention assessment, and the operating expenses of the local Chapter. An additional fee may be required from all new initiates and members to attend the end of the semester banquet.
	\subsection{Operating Budget}
	Within two weeks of the beginning of the semester, the Treasurer, with advice from other officers, shall prepare an operating budget for the semester. The budget shall be submitted to the Chapter for approval by a majority vote at the immediate next meeting of the semester. The budget shall include a recommended amount for the initiation fee to be charged during the year. Any additional expenses not provided for by the adopted budget must be approved by the Chapter officers, except that the Treasurer shall be authorized to advance to the Convention delegate a sum sufficient to cover expenses he or she expects to incur in attending the annual Convention of the Association.
	\subsection{Minimum Balance}
	There shall be at all times a balance of at least \$1000.00 in the Chapter treasury.
	\subsection{Budget Modifications}
	The Treasurer may modify the semester budget to account for unexpected expenses. All modifications to the budget require a majority vote of all elected officers.
	\subsection{Budget Expenditures}
	Budget expenditures shall be made by check, signed by the Treasurer, or by the President, or by the Head Advisor. Other electronic methods may be used with approval of the Advisory Board. Cash expenditures may be made subject to approval of the Advisory Board. Any non-budget expenditures require a majority vote of all elected officers. %Changed per Matt's suggestions
	\subsection{Financial Report}
	Immediately following the end of the academic year, the outgoing Treasurer shall prepare a financial report to be presented to the Treasurer-elect and the Advisory Board.
	\subsection{Net Earnings}
	No part of the net earnings of the Chapter will go to the benefit of, or be distributable to, members or officers of the Chapter or to any other individual.
	\subsection{Execution of Instruments}
	The board of directors, except as otherwise provided in these bylaws, may by resolution authorize any officer or agent of the corporation to enter into any contract or execute and deliver any instrument in the name of and on behalf of the corporation, and such authority may be general or confined to specific instances. Unless so authorized, no officer, agent, or employee shall have any power or authority to bind the corporation by any contract or engagement or to pledge its credit or to render it liable monetarily for any purpose or in any amount.
	\subsection{Checks and Notes}
	Except as otherwise specifically determined by resolution of the board of directors, or as otherwise required by law, checks, drafts, promissory notes, orders for the payment of money, and other evidence of indebtedness of the corporation shall be signed by the Treasurer and countersigned by the President of the corporation.
	\subsection{Deposits}
	All funds of the corporation shall be deposited from time to time to the credit of the corporation in such banks, trust companies, or other depositories as the board of directors may select.
	\subsection{Gifts}
	The board of directors may accept on behalf of the corporation any contribution, gift, bequest, or devise for the charitable or public purposes of this corporation.
	
	\section{Discipline}
	\subsection{Seriousness}
	It is the intent of this disciplinary Bylaw to impress the membership with the seriousness of purpose of this Chapter and of The Tau Beta Pi Association, and to enable the officers and dedicated members of this Chapter to accomplish necessary business in a manner not inconvenienced or encumbered by a lack of interest on the part of a few members.
	\subsection{Guidelines}
	Discipline shall be in accordance with C-IX, 4\&5.
	\subsection{Punctuality}
	A system of fines for habitual absence or tardiness may be established each year in which it is desired by the Chapter. A majority vote of the active membership shall be required to establish this system of fines for a period of one year.
	
	\section{Corporate Records, Records, and Seal}
	\subsection{Records}
	All permanent records of the Chapter shall be kept current and up-to- date.
	\subsection{Initiation Equipment}
	The initiation equipment shall be maintained in good order and in a secure manner by the Vice President, and the Ritual and its related materials shall be kept up-to-date and under lock and key when not in use.
	\subsection{Transfer of Records}
	All records and an inventory of all physical equipment owned by the Chapter shall be turned over to the new officers at the Installation of Officers Meeting.
	\subsection{Display of Charter}
	The Charter of this Chapter shall be prominently displayed at a location determined by the Dean of Engineering.
	\subsection{Availability of Records for Inspection}
	All records of this Chapter shall be open for inspection to any member of the Association and to any official of the University who has received approval from the Advisory Board, except that the Ritual may not be inspected by non-members of the Association.
	\subsection{Maintenance of Corporate Records}
	The corporation shall keep at its principal office in the State of California:
	\begin{enumerate}[\indent (a)]
		\item Minutes of all meetings of directors, committees of the board and, if this corporation has members, of all meetings of members, indicating the time and place of holding such meetings, whether regular or special, how called, the notice given, and the names of those present and the proceedings thereof;
		\item Adequate and correct books and records of account, including accounts of its properties and business transactions and accounts of its assets, liabilities, receipts, disbursements, gains, and losses;
		\item A record of its members, if any, indicating their names and addresses and, if applicable, the class of membership held by each member and the termination date of any membership;
		\item A copy of the corporation\textquotesingle s articles of incorporation and bylaws as amended to date, which shall be open to inspection by the members, if any, of the corporation at all reasonable times during office hours.
	\end{enumerate}
	\subsection{Corporate Seal}
	The board of directors may adopt, use, and at will alter, a corporate seal. Such seal shall be kept at the principal office of the corporation. Failure to affix the seal to corporate instruments, however, shall not affect the validity of any such instrument. The Corporation is not required to have a seal.
	\subsection{Directors' Inspection Rights}
	Every director shall have the absolute right at any reasonable time to inspect and copy all books, records, and documents of every kind and to inspect the physical properties of the corporation.
	\subsection{Members' Inspection Rights}
	If this corporation has any members, then each and every member shall have the following inspection rights, for a purpose reasonably related to such person\textquotesingle s interest as a member:
	\begin{enumerate}[\indent (a)]
		\item To inspect and copy the record of all members\textquotesingle  names, addresses, and voting rights, at reasonable times, upon five (5) business days prior written demand on the corporation, which demand shall state the purpose for which the inspection rights are requested.
		\item To obtain from the secretary of the corporation, upon written demand and payment of a reasonable charge, an alphabetized list of the names, addresses, and voting rights of those members entitled to vote for the election of directors as of the most recent record date for which the list has been compiled or as of the date specified by the member subsequent to the date of demand. The demand shall state the purpose for which the list is requested. The membership list shall be made available on or before the later of ten (10) business days after the demand is received or after the date specified therein as of which the list is to be compiled.
		\item To inspect at any reasonable time the books, records, or minutes of proceedings of the members or of the board or committees of the board, upon written demand on the corporation by the member, for a purpose reasonably related to such person\textquotesingle s interests as a member.
	\end{enumerate}
	\subsection{Right to Copy and Make Extracts}
	Any inspection under the provisions of this Article may be made in person or by agent or attorney and the right to inspection includes the right to copy and make extracts.
	\subsection{Annual Report}
	The board shall cause an annual report to be furnished, by mail or via electronic delivery, not later than one hundred and twenty (120) days after the close of the corporation’s fiscal year to all directors of the corporation and, if this corporation has members, to any member who requests it in writing, which report shall contain the following information in appropriate detail:
	\begin{enumerate}[\indent (a)]
		\item The assets and liabilities, including the trust funds, of the corporation as of the end of the fiscal year;
		\item The principal changes in assets and liabilities, including trust funds, during the fiscal year;
		\item The revenue or receipts of the corporation, both unrestricted and restricted to particular purposes, for the fiscal year;
		\item The expenses or disbursements of the corporation, for both general and restricted purposes, during the fiscal year;
		\item Any information required by Section 12 of this Article.
	\end{enumerate}
	The annual report shall be accompanied by any report thereon of independent accountants, or, if there is no such report, the certificate of an authorized officer of the corporation that such statements were prepared without audit from the books and records of the corporation.
	If this corporation has members, then, if this corporation receives Twenty-Five Thousand Dollars (\$25,000), or more, in gross revenues or receipts during the fiscal year, this corporation shall automatically send the above annual report to all members, in such manner, at such time, and with such contents, including an accompanying report from independent accountants or certification of a corporate officer, as specified by the above provisions of this Section relating to the annual report.
	
	\subsection{Annual Statement of Specific Transactions to Members}
	This corporation shall, within one hundred and twenty (120) days after the close of its fiscal year, mail or send via electronic delivery to all directors and any and all members a statement that briefly describes the amount and circumstances of any indemnification or transaction of the following kind, if any such indemnification or transactions have occurred:
	
	Any transaction in which the corporation, or its parent or its subsidiary, was a party, and in which either of the following had a direct or indirect material financial interest:
	\begin{enumerate}[\indent (a)]
		\item Any director or officer of the corporation, or its parent or its subsidiary (a mere common directorship shall not be considered a material financial interest); or
		\item Any holder of more than ten percent (10\%) of the voting power of the corporation, its parent, or its subsidiary.
	\end{enumerate}
	The above statement need only be provided with respect to a transaction during the previous fiscal year involving more than Fifty Thousand Dollars (\$50,000) or which was one of a number of transactions with the same persons involving, in the aggregate, more than Fifty Thousand Dollars (\$50,000).
	Similarly, the statement need only be provided with respect to indemnifications or advances aggregating more than Ten Thousand Dollars (\$10,000) paid during the previous fiscal year to any director or officer, except that no such statement need be made if such indemnification was approved by the members pursuant to Section 5238(e)(2) of the California Nonprofit Public Benefit Corporation Law.
	Any statement required by this Section shall briefly describe the names of the interested persons involved in such transactions, stating each person\textquotesingle s relationship to the corporation, the nature of such person\textquotesingle s interest in the transaction, and, where practical, the amount of such interest, provided that in the case of a transaction with a partnership of which such person is a partner, only the interest of the partnership need be stated.
	If this corporation has any members and provides all members with an annual report according to the provisions of Section 6 of this Article, then such annual report shall include the information required by this Section.
	\subsection{Fiscal Year of the Corporation}
	The fiscal year of the corporation shall begin on the first day June and end on the last day of May in each year.
	
	\section{Conflict of Interest and Compensation Approval Policies}
	\subsection{Purpose of Conflict of Interest Policy}
	The purpose of this conflict of interest policy is to protect this tax ­exempt corporation\textquotesingle s interest when it is contemplating entering into a transaction or arrangement that might benefit the private interest of an officer or director of the corporation or any “disqualified person” as defined in Section 4958(f)(1) of the Internal Revenue Code and as amplified by Section 53.4958-3 of the IRS Regulations and which might result in a possible "excess benefit transaction" as defined in Section 4958(c)(l)(A) of the Internal Revenue Code and as amplified by Section 53.4958 of the IRS Regulations. This policy is intended to supplement but not replace any applicable state and federal laws governing conflict of interest applicable to nonprofit and charitable organizations.
	\subsection{Definitions}
		\subsubsection{Interested Person}
		Any director, principal officer, member of a committee with governing board delegated powers, or any other person who is a ``disqualified person'' as defined in Section 4958(f)(l) of the Internal Revenue Code and as amplified by Section 53.4958-3 of the IRS Regulations, who has a direct or indirect financial interest, as defined below, is an interested person.
		\subsubsection{Financial Interest}
		A person has a financial interest if the person has, directly or indirectly, through business, investment, or family:
		\begin{enumerate}[\indent (a)] 
			\item an ownership or investment interest in any entity with which the corporation has a transaction or arrangement;
			\item a compensation arrangement with the corporation or with any entity or individual with which the corporation has a transaction or arrangement; or
			\item a potential ownership or investment interest in, or compensation arrangement with, any entity or individual with which the corporation is negotiating a transaction or arrangement.
		\end{enumerate}
		Compensation includes direct and indirect remuneration as well as gifts or favors that are not insubstantial.
		
		A financial interest is not necessarily a conflict of interest. Under Section 3, Clause 2, of this Article, a person who has a financial interest may have a conflict of interest only if the appropriate governing board or committee decides that a conflict of interest exists.
	
	
	\subsection{Conflict of Interest Avoidance Procedures}
		\subsubsection{Duty to Disclose}
			In connection with any actual or possible conflict of interest, an interested person must disclose the existence of the financial interest and be given the opportunity to disclose all material facts to the directors and members of committees with governing board delegated powers considering the proposed transaction or arrangement.
		\subsubsection{Determining Whether a Conflict of Interest Exists}
		After disclosure of the financial interest and all material facts, and after any discussion with the interested person, he/she shall leave the governing board or committee meeting while the determination of a conflict of interest is discussed and voted upon. The remaining board or committee members shall decide if a conflict of interest exists.
		\subsubsection{Procedures for Addressing the Conflict of Interest}
		An interested person may make a presentation at the governing board or committee meeting, but after the presentation, he/she shall leave the meeting during the discussion of, and the vote on, the transaction or arrangement involving the possible conflict of interest.
		
		The chairperson of the governing board or committee shall, if appropriate, appoint a disinterested person or committee to investigate alternatives to the proposed transaction or arrangement.
		
		After exercising due diligence, the governing board or committee shall determine whether the corporation can obtain with reasonable efforts a more advantageous transaction or arrangement from a person or entity that would not give rise to a conflict of interest.
		
		If a more advantageous transaction or arrangement is not reasonably possible under circumstances not producing a conflict of interest, the governing board or committee shall determine by a majority vote of the disinterested directors whether the transaction or arrangement is in the corporation\textquotesingle s best interest, for its own benefit, and whether it is fair and reasonable. In conformity with the above determination, it shall make its decision as to whether to enter into the transaction or arrangement.
		
		\subsubsection{Violations of the Conflicts of Interest Policy}
		If the governing board or committee has reasonable cause to believe a member has failed to disclose actual or possible conflicts of interest, it shall inform the member of the basis for such belief and afford the member an opportunity to explain the alleged failure to disclose.
		
		If, after hearing the member\textquotesingle s response and after making further investigation as warranted by the circumstances, the governing board or committee determines the member has failed to disclose an actual or possible conflict of interest, it shall take appropriate disciplinary and corrective action such as a penalty, censure, or expulsion.
	\subsection{Records of Board and Board Committee Proceedings}
	The minutes of meetings of the governing board and all committees with board delegated powers shall contain:
	\begin{enumerate}[\indent (a)] 
		\item The names of the persons who disclosed or otherwise were found to have a financial interest in connection with an actual or possible conflict of interest, the nature of the financial interest, any action taken to determine whether a conflict of interest was present, and the governing board\textquotesingle s or committee\textquotesingle s decision as to whether a conflict of interest in fact existed.
		\item The names of the persons who were present for discussions and votes relating to the transaction or arrangement, the content of the discussion, including any alternatives to the proposed transaction or arrangement/ and a record of any votes taken in connection with the proceedings.
	\end{enumerate}
	
	\subsection{Compensation Approval Policies}
	A voting member of the governing board who receives compensation directly or indirectly, from the corporation for services is precluded from voting on matters pertaining to that member\textquotesingle s compensation.
	
	A voting member of any committee whose jurisdiction includes compensation matters and who receives compensation, directly or indirectly, from the corporation for services is precluded from voting on matters pertaining to that member\textquotesingle s compensation.
	
	No voting member of the governing board or any committee whose jurisdiction includes compensation matters and who receives compensation/ directly or indirectly/ from the corporation, either individually or collectively, is prohibited from providing information to any committee regarding compensation.
	
	When approving compensation for directors, officers and employees, contractors, and any other compensation contract or arrangement, in addition to complying with the conflict of interest requirements and policies contained in the preceding and following sections of this article as well as the preceding paragraphs of this section of this article, the board or a duly constituted compensation committee of the board shall also comply with the following additional requirements and procedures:
	
	\begin{enumerate}[\indent (a)] 
		\item the terms of compensation shall be approved by the board or compensation committee prior to the first payment of compensation.
		\item all members of the board or compensation committee who approve compensation arrangements must not have a conflict of interest with respect to the compensation arrangement as specified in IRS Regulation Section 53.4958-G(c)(iii), which generally requires that each board member or committee member approving a compensation arrangement between this organization and a "disqualified person" (as defined in Section 4958(f)(l) of the Internal Revenue Code and as amplified by Section 53.4958-3 of the IRS Regulations):
		\begin{enumerate}[\indent 1)] 
			\item is not the person who is the subject of compensation arrangement, or a family member of such person;
			\item is not in an employment relationship subject to the direction or control of the person who is the subject of compensation arrangement
			\item does not receive compensation or other payments subject to approval by the person who is the subject of compensation arrangement
			\item has no material financial interest affected by the compensation arrangement; and
			\item does not approve a transaction providing economic benefits to the person who is the subject of the compensation arrangement/ who in turn has approved or will approve a transaction providing benefits to the board or committee member.
		\end{enumerate}
		\item The board or compensation committee shall obtain and rely upon appropriate data as to comparability prior to approving the terms of compensation. Appropriate data may include the following:
		\begin{enumerate}[\indent 1)] 
			\item compensation levels paid by similarly situated organizations, both taxable and tax-exempt, for functionally comparable positions. "Similarly situated" organizations are those of a similar size and purpose and with similar resources
			\item the availability of similar services in the geographic area of this organization
			\item current compensation surveys compiled by independent firms
			\item actual written offers from similar institutions competing for the services of the person who is the subject of the compensation arrangement.
		\end{enumerate}
		As allowed by IRS Regulation 4958-6, if this organization has average annual gross receipts (including contributions) for its three prior tax years of less than \$1 million, the board or compensation committee will have obtained and relied upon appropriate data as to comparability if it obtains and relies upon data on compensation paid by three comparable organizations in the same or similar communities for similar services.
		\item the terms of compensation and the basis for approving them shall be recorded in written minutes of the meeting of the board or compensation committee that approved the compensation. Such documentation shall include:
		\begin{enumerate}[\indent 1)] 
			\item the terms of the compensation arrangement and the date it was approved
			\item the members of the board or compensation committee who were present during debate on the transaction/ those who voted on it, and the votes cast by each board or committee member 
			\item the comparability data obtained and relied upon and how the data was obtained.
			\item If the board or compensation committee determines that reasonable compensation for a specific position in this organization or for providing services under any other compensation arrangement with this organization is higher or lower than the range of comparability data obtained/ the board or committee shall record in the minutes of the meeting the basis for its determination.
			\item If the board or committee makes adjustments to comparability data due to geographic area or other specific conditions, these adjustments and the reasons for them shall be recorded in the minutes of the board or committee meeting.
			\item Any actions taken with respect to determining if a board or committee member had a conflict of interest with respect to the compensation arrangement, and if so, actions taken to make sure the member with the conflict of interest did not affect or participate in the approval of the transaction. For example, a notation in the records that after a finding of conflict of interest by a member, the member with the conflict of interest was asked to, and did, leave the meeting prior to a discussion of the compensation arrangement and a taking of the votes to approve the arrangement shall be recorded.
			\item The minutes of board or committee meetings at which compensation arrangements are approved must be prepared before the later of the date of the next board or committee meeting or 60 days after the final actions of the board or committee are taken with respect to the approval of the compensation arrangements. The minutes must be reviewed and approved by the board and committee as reasonable, accurate, and complete within a reasonable period thereafter, normally prior to or at the next board or committee meeting following final action on the arrangement by the board or committee.
			\end{enumerate}
		\end{enumerate}
	\subsection{Annual Statements}
	Each director, principal officer, and member of a committee with governing board-delegated powers shall annually sign a statement that affirms such person:
	\begin{enumerate}[\indent (a)] 
		\item has received a copy of the conflicts of interest policy,
		\item has read and understands the policy,
		\item has agreed to comply with the policy, and
		\item understands the corporation is charitable and in order to maintain its federal tax exemption it must engage primarily in activities which accomplish one or more of its tax-exempt purposes.
	\end{enumerate}
	
	\subsection{Periodic Review}
	To ensure the corporation operates in a manner consistent with charitable purposes and does not engage in activities that could jeopardize its tax-exempt status, periodic reviews shall be conducted. The periodic reviews shall, at a minimum, include the following subjects:
	\begin{enumerate}[\indent (a)] 
		\item Whether compensation arrangements and benefits are reasonable, based on competent survey information, and the result of arm\textquotesingle s-length bargaining.
		\item Whether partnerships, joint ventures, and arrangements with management organizations conform to the corporation\textquotesingle s written policies, are properly recorded, reflect reasonable investment or payments for goods and services, further charitable purposes, and do not result in inurement, impermissible private benefit, or in an excess benefit transaction.
	\end{enumerate}
	
	\subsection{Use of Outside Experts}
	When conducting the periodic reviews as provided for in Section 7, the corporation may, but need not, use outside advisors. If outside experts are used, their use shall not relieve the governing board of its responsibility for ensuring periodic reviews are conducted.
	
	\section{Amendment of Bylaws}
	\subsection{Proposal for Amendments}
	Subject to any provision of law applicable to the amendment of bylaws of public benefit nonprofit corporations, amendments to these Bylaws may be proposed by any three active members of the Chapter. A proposed amendment shall be submitted in writing to the President and signed by the members proposing it.
	\subsection{Quorum}
	These Bylaws may be amended by a three-fourths affirmative vote of the active membership of the Chapter, subject to the approval of the Advisory Board, as provided in C-VI, 7(b).
	\subsection{Recording of Amendments}
	The Corresponding Secretary shall send a copy of the Bylaws as amended, to the Secretary-Treasurer of the Association within two weeks after an amendment is adopted. All amendments, additions or deletions to this document must be filed with the LEAD Center in 432 Eshleman Hall.
	
	\section{Suspension of Bylaws}
	\subsection{Quorum}
	These Bylaws may be suspended only by a three-fourths affirmative vote of the active membership of the Chapter and by a four-sevenths affirmative vote of the Advisory Board, as provided in C-VI, 5.
	
	\section{Amendment of Articles}	
	\subsection{Amendment of Articles Before Admission of Members}
	Before any members have been admitted to the corporation, any amendment of the articles of incorporation may be adopted by approval of a three-quarters majority vote of the board of directors.
	\subsection{Amendment of Articles After Admission of Members}
	After members, if any, have been admitted to the corporation, amendment of the articles of incorporation may be adopted by the approval of a three-quarters majority vote of the board of directors and by the approval of a two-thirds majority vote of the members of this corporation.
	\subsection{Certain Amendments}
	Notwithstanding the above sections of this Article, this corporation shall not amend its articles of incorporation to alter any statement which appears in the original articles of incorporation of the names and addresses of the first directors of this corporation, nor the name and address of its initial agent, except to correct an error in such statement or to delete such statement after the corporation has filed a "Statement by a Domestic Nonprofit Corporation" pursuant to Section 6210 of the California Nonprofit Corporation Law.
	
	\section{Prohibition Against Sharing Corporate Profits and Assets}
	\subsection{Prohibition Against Sharing Corporate Profits and Assets}
	No member, director, officer, employee, or other person connected with this corporation, or any private individual, shall receive at any time any of the net earnings or pecuniary profit from the operations of the corporation, provided, however, that this provision shall not prevent payment to any such person of reasonable compensation for services performed for the corporation in effecting any of its public or charitable purposes, provided that such compensation is otherwise permitted by these bylaws and is fixed by resolution of the board of directors; and no such person or persons shall be entitled to share in the distribution of, and shall not receive any of the corporate assets on dissolution of the corporation. All members, if any, of the corporation shall be deemed to have expressly consented and agreed that on such dissolution or winding up of the affairs of the corporation, whether voluntarily or involuntarily, the assets of the corporation, after all debts have been satisfied, shall be distributed as required by the articles of incorporation of this corporation and not otherwise.
	
	\section{Dissolution}
	\subsection{Proposal for Dissolution}
	Dissolution of the Chapter may be proposed by any three active members of the Chapter. A proposal for dissolution shall be submitted in writing to the President and signed by the members proposing it.
	\subsection{Approval of Dissolution}
	Dissolution must be approved by a three-fourths affirmative vote of the active membership of the Chapter, subject to the approval of the Advisory Board, as provided in C-VI, 7(b).
	\subsection{Distribution of Funds}
	In the event of dissolution of the Chapter, the residual assets shall be distributed to The Tau Beta Pi Association, Inc., a corporation organized and operated exclusively for educational and scientific purposes and exempt from federal income tax under Section 501(c)(3) of the U.S. Internal Revenue Code of 1954. Any of such assets not so disposed of shall be distributed to a Federal, State, or Local Government for public purposes.
	\subsection{Returning of ASUC Funds}
	All unspent ASUC funds shall remain the property of the ASUC; all Graduate Assembly funds shall remain the property of the Graduate Assembly. After 5 or more years of inactivity, any privately obtained funds left in accounts will be donated to the ASUC carry forward account.
	\subsection{Clarifications to Distribution of Funds}
	The property of this Corporation is irrevocably dedicated to charitable purposes and no part of the net income or assets of the corporation shall ever inure to the benefit of any director, trustee, member or officer of this corporation, or to any private person. 
	
	Upon the dissolution or winding up of the corporation, any assets remaining after payment of, or provision for payment of, all debts and liabilities shall be distributed to a governmental entity described in Section 170(b)(1) (A)(v) of the Internal Revenue Code, or to a nonprofit fund, foundation, or corporation which is organized and operated exclusively for charitable purposes, which has established its tax exempt status under Section 501(c)(3) of the Internal Revenue Code, and which is qualified to receive ``qualified conservation contributions" within the meaning of Section 170(h) of said Code, or the corresponding provisions of any future statute of the United States.
	
	In the event of a liquidation of this corporation, all corporate assets shall be disposed of in such a manner as may be directed by decree of the superior court for the county in which the corporation has its principal office, on petition therefore by the Attorney General of by any person concerned in the dissolution, in a proceeding to which the Attorney General is a party.
	\pagebreak
\section*{Written Consent of Directors Adopting Bylaws}
	We, the undersigned, are all of the persons named as the initial directors in the articles of incorporation of Tau Beta Pi CA-Alpha, Inc., a California nonprofit corporation, and, pursuant to the authority granted to the directors by these bylaws to take action by unanimous written consent without a meeting, consent to, and hereby do, adopt the foregoing bylaws pages as the bylaws of this corporation.
	
	\vspace{1 cm}
	\noindent \makebox[\textwidth][l]{
		\makebox[9cm][l] {\hrulefill} \hfill 
		\makebox[5cm][l] {\hrulefill} 
	}
	\noindent \makebox[\textwidth][l]{
		Yu-Han Chen, Director  \hfill Date
	}
	
	\vspace{1 cm}
	\noindent \makebox[\textwidth][l]{
		\makebox[9cm][l] {\hrulefill} \hfill 
		\makebox[5cm][l] {\hrulefill} 
	}
	\noindent \makebox[\textwidth][l]{
		Xiao-Yu Fu, Director  \hfill Date
	}
	
	\vspace{1cm}
	\noindent \makebox[\textwidth][l]{
		\makebox[9cm][l] {\hrulefill} \hfill 
		\makebox[5cm][l] {\hrulefill} 
	}
	\noindent \makebox[\textwidth][l]{
		Dennis K. Lieu, Director  \hfill Date
	}
	\pagebreak
	\section*{Certificate}
	This is to certify that the foregoing is a true and correct copy of the bylaws of the corporation named in the title thereto and that such bylaws were duly adopted by the board of directors of said corporation on the date set forth below.
	
	\vspace{1cm}
	\noindent \makebox[\textwidth][l]{
		\makebox[9cm][l] {\hrulefill} \hfill 
		\makebox[5cm][l] {\hrulefill} 
	}
	\noindent \makebox[\textwidth][l]{
		Anna Garachtchenko, Acting Secretary  \hfill Date
	}
	
	\pagebreak
	\section*{Waiver of Notice and Consent to Holding of First Meeting of Board of Directors}
		\begin{large}
			\uppercase{Tau Beta Pi CA-Alpha, Inc.} \\
			A California Nonprofit Public Benefit Corporation\\
		\end{large} 
		
		\noindent We, the undersigned, being a majority of the directors of Tau Beta Pi CA-Alpha, Inc., a California nonprofit public benefit corporation, hereby waive notice of the first meeting of the board of directors of the corporation and consent to the holding of said meeting at Tau Beta Pi, Engineering Student Services, 230 Bechtel at 9:00 A.M., and consent to the transaction of any and all business by the directors at the meeting, including, without limitation, the adoption of bylaws, the election of officers, and the selection of the place where the corporation\textquotesingle s bank account will be maintained.
	
	
	\vspace{1 cm}
	\noindent \makebox[\textwidth][l]{
		\makebox[9cm][l] {\hrulefill} \hfill 
		\makebox[5cm][l] {\hrulefill} 
	}
	\noindent \makebox[\textwidth][l]{
		Yu-Han Chen, Director  \hfill Date
	}
	
	\vspace{1 cm}
	\noindent \makebox[\textwidth][l]{
		\makebox[9cm][l] {\hrulefill} \hfill 
		\makebox[5cm][l] {\hrulefill} 
	}
	\noindent \makebox[\textwidth][l]{
		Xiao-Yu Fu, Director  \hfill Date
	}
	
	\vspace{1cm}
	\noindent \makebox[\textwidth][l]{
		\makebox[9cm][l] {\hrulefill} \hfill 
		\makebox[5cm][l] {\hrulefill} 
	}
	\noindent \makebox[\textwidth][l]{
		Dennis K. Lieu, Director  \hfill Date
	}
	\pagebreak
	\section*{Minutes of First Meeting of Board of Directors}
	\begin{large}
		\uppercase{Tau Beta Pi CA-Alpha, Inc.} \\
		A California Nonprofit Public Benefit Corporation\\
	\end{large} 
	
	\noindent The board of directors of Tau Beta Pi CA-Alpha, Inc. held its first meeting on April 7, 2015 at Tau Beta Pi, Engineering Student Services, 230 Bechtel. Written waiver of notice was signed by all of the directors.\\
	
	\noindent The following directors, constituting a quorum of the full board, were present at the meeting: Yu-Han Chen, Dennis K. Lieu, Xiao-Yu Fu.\\
	
	\noindent Anna Garachtchenko acted as chairperson secretary and then presided over the meeting.\\
	
	\noindent The meeting was held pursuant to written waiver of notice.
	
	\subsection*{Bylaws}
	There was then presented to the meeting for adoption a proposed set of bylaws of the corporation. The bylaws were considered and discussed and, on motion duly made and seconded, it was unanimously approved.
	
	The bylaws presented at the meeting were adopted as the bylaws of the corporation.
	
	\subsection*{California and Federal Tax Exemptions}
	The Board directed Michael Chen to prepare and submit to the California Franchise Tax Board and to the Internal Revenue Service, applications for exemptions from payment of state corporate franchise taxes and payment of federal corporate income.
	\subsection*{Election of Officers}
	The chairperson then announced that the next item of business was the election of officers. Upon motion, the following persons were unanimously elected to the offices shown before their names:
	\begin{tabbing}
		{\hskip 1in} \= \kill
		President\>	Cameron Bates \\
		Secretary \> Anna Garachtchenko \\
		Treasurer \> Michael Chen 
	\end{tabbing}
	\subsection*{Compensation of Officers}
	There followed a discussion concerning the compensation to be paid by the corporation to its officers. Upon motion duly made and seconded, it was unanimously RESOLVED, that the officers of this corporation shall serve without compensation
	\subsection*{Corporate Seal}
	The Board decided the Corporation would not have a seal at this time.
	\subsection*{Principal Office}
	The principal office for the transaction of business of the corporation shall be at Tau Beta Pi, Engineering Student Services, 230 Bechtel, in Berkeley, California.
	
	\subsection*{Bank Account}
	The Treasurer of this corporation and Directors were authorized and directed to establish an account with a bank and to deposit the funds of this corporation therein. Both were to report back to the Board when the bank account was established.
	
	It was decided, that all checks, drafts, and other instruments obligating this corporation to pay money shall be signed on behalf of this corporation by the Treasurer and Directors.
	
	It was further decided, that said bank be and hereby is authorized to honor and pay all checks and drafts of this corporation signed as provided herein.\\
	
	\noindent Since there was no further business to come before the meeting, the meeting was adjourned.
	
	\vspace{1cm}
	\noindent \makebox[\textwidth][l]{
		\makebox[9cm][l] {\hrulefill} \hfill 
		\makebox[5cm][l] {\hrulefill} 
	}
	\noindent \makebox[\textwidth][l]{
		Anna Garachtchenko, Acting Secretary  \hfill Date
	}
	
\end{document}