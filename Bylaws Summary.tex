\documentclass{article}
\usepackage[letterpaper, margin = 1 in]{geometry}
\usepackage{enumerate}
\usepackage{paralist}
\usepackage[english]{babel}
\usepackage{csquotes}
\usepackage{textcomp}
\begin{document}
	\pagenumbering{gobble} 
	\title{\textbf {\uppercase{Summary of Bylaws}}}
	\author{\uppercase{Tau Beta Pi CA-Alpha, Inc.} \\ A California Nonprofit Public Benefit Corporation}
	\date{November 11, 2016}
	\maketitle
	\clearpage
	
	\pagenumbering{arabic}
	
	\section{Associations}
	This section defines our relations with the Tau Beta Pi Association, Inc. (i.e. "Nationals"), and the University of California at Berkeley. We will follow local regulations at the university before those of Nationals, in case a conflict between them were to occur.
	
	\section{Offices}
	This section defines the TBP CA-A, Inc.'s main office as 101 O'Brien Hall, and provides a method for changing the office.
	
	\section{Purposes}
	This section defines the purpose of the corporation as the following:
	\blockquote{The primary objectives and purposes of this corporation shall be to recognize, celebrate, and champion excellence and integrity in engineering at the University of California at Berkeley, and to encourage and promote the engineering discipline in the community and beyond. To do so, the Chapter will host professional development events, organize industry opportunities, provide social engagements and activities for the engineering community, and encourage outreach and public service to promote engineering and give back to the community.}
	The section also defines the "non-profit" nature of the corporation as required by IRS code and California law.
	
	\section{Directors}
	This section defines the powers and responsibility of the Board of Directors, as well as provides . Other than responsibilities as defined by California Law (e.g. registering the company), the Board of Directors transfers its day-to-day management activities to the Executive Committee, as per Article 6, Section 1.
	
	\section {Officers and Members}
	This section defines the officer/advisor structure of the Chapter and Corporation, active and inactive membership, and the Advisory Board. 
	\subsection {Officer-Related Highlights:}
	\begin{itemize}
		\item As for officers, the chapter shall consist of at minimum an \textbf{Executive Board}. This consists of the President, Vice President(s), Recording Secretary, Corresponding Secretary, and Treasurer.
		\item The \textbf{Advisory Board} shall be an "advisory and judiciary" committee that serves as the Chapter's conscience. It consists of the President, first Vice President, Corresponding Secretary, and four Alumnus Advisors.
		\item The Chapter may define additional officers, subject for approval by the Advisory Board.
		\item \textbf{Corporate Officers} are different from Chapter Officers and report directly to the Board of Directors. Currently, corporate officers consist of the CEO, CFO, and Secretary, all of which are necessary for incorporation in the State of California.
		\item A process involving the Advisory Board exists to remove/impeach Chapter Officers. Corporate Officers are appointed and dismissed by the Board of Directors.
	\end{itemize}
	
	\subsection {Member-Related Highlights:}
	\begin{itemize}
		\item Members of TBP CA-A are categorized as "active" or "inactive" based on their attendance at three key Chapter meetings, and only active members can vote or hold office.
		\item Only active members who are also currently enrolled students at UC Berkeley can vote on new members and on the amounts of the initiation fee, dues, fines, and assessments.
		\item "Temporary active status" is granted to members who are track to achieve "active status" by the conclusion of the semester, which will allow them to be "active" for the purpose of "official chapter business."
		\item The Advisory Board may designate certain active members as "temporarily inactive" through four-seventh majority vote in order to conduct business at a meeting. This designation will expire by the end of the meeting.
		\item The Chapter will use the number of members who obtained "active status" by the end of the previous semester as a fixed number for calculating quorum.
	\end{itemize}
	
	\section{Committees}
	This section grants the Executive Committee powers to manage the Chapter on a day-to-day business, with the exception of certain corporate and bylaw-related activities. It also provides a method for the Chapter to create additional committees, subject to approval by the Advisory Board.
	
	\section{Chapter Meetings}
	This section defines the different types of meetings that the Chapter must hold, as well as the procedures for holding each meeting.
	
	\subsection{Procedural Highlights}
	\begin{itemize}
		\item Every semester, the chapter must hold an introductory meeting (i.e. Candidate Meeting), an Election of Candidate meeting, a Final Candidate Review meeting, a Candidate Initiation meeting, and an Election of Officers meeting.
		\item Officer meetings must be held weekly, and attendance is mandatory.
		\item Advisory Board meetings must be held on campus once a year. Special conditions govern meetings held by teleconference.
		\item Elections and Final Candidate Review Meetings both require a 14-day notice to all active members. 30\% of active members may object to a meeting's time, location, or purpose by sending a request to the President 48-hours in advance of the meeting.
		\item Meetings, with the exception of Elections and Final Candidate Review, must conclude within two hours.
		\item The chapter must provide a method for active members to participate in meetings remotely.
	\end{itemize}
	\subsection{Quorum}
	\begin{itemize}
		\item Quorum for advisory board meetings is four-seventh of the Advisory Board.
		\item Quorum for meetings for "routine chapter business" shall be one-half of active members.
		\item Quorum for Election of Candidates, Final Candidate Review, and approval nationals-related business will be three-fourth of active members.
	\end{itemize}
	
	\section{Election and Initiation of New Members}
	This section describes the process and requirements for election new members.
	
	The Chapter must first hold an "Election of Candidates meeting," where active members accept new candidates for membership. Later in the semester, the Chapter will hold a "Final Candidate Review," where active members re-evaluate and possibly revoke electees. Electees ultimately become members of TBP through initiation.
	
	\section{Election of Officers}
	This section describes the process for election Chapter officers.
	\subsection*{Highlights:}
	\begin{itemize}
		\item Elections for each semester must be held within two weeks of new member initiation.
		\item The order and procedure of elections, as well as post-election officer transitions, are described in the President's Book.
		\item The quorum for elections shall be one-half of active members.
		\item A majority of quorum is required for the election of any officer, and runoff elections shall be held until one candidate receives a majority of votes.
		\item Special elections shall be held to fill any vacancies that occur between elections.
		\item Resignations must be tended to the Chapter President in writing, and will be effective seven days from the date of submission.
	\end{itemize}
	\section{Execution of Instruments, Deposits, and Funds}
	This section defines the procedure for setting and changing initiation fees, the operating budget, deposits, expenditures, and related reports.
	
	\subsection*{Highlights:}
	\begin{itemize}
		\item The Treasurer will prepare a semesterly budget in the beginning of every semester, which will then be subject for approval by a majority vote at the next officer meeting.
		\item Extra-budget expenses and changes to the budget are subject to approval by Chapter officers.
		\item The President, Treasurer, and Head Advisor, may sign checks that represent expenditures of the Chapter. All electronic expenditure methods must be subject to approval by the Advisory Board
	\end{itemize}
	
	\section{Discipline}
	This section describes the seriousness in discipline expected from all members of Tau Beta Pi, and provides a procedure for the President to set fines for members who break discipline.
	
	\section{Corporate Records, Report, and Seal}
	This section describes the record-keeping procedures for the Chapter, directors' and members' rights to inspect records, and the necessary annual reports and statements that the Chapter must provide to all members.
	
	\section{Conflict of Interest and Compensation Approval Policies}
	This section defines the conflict-of-interest policies of the Corporation, as well as identification and avoidance procedures. It also defines the process for compensating members of the governing Board of Directors. Corporate officers and Board of Director members must sign a statement acknowledging this policy on an annual basis. The Corporation must also conduct periodic reviews.	
	
	\section {Amendment/Suspension of Bylaws and Articles of Incorporation}
	Sections 14 - 16 describes the process needed to alter the Bylaws or the Articles of Incorporation.
	
	A three-fourth affirmative vote by all active members, four-seventh affirmative vote by the Advisory Board, and a three-quarters majority vote by the Board of Directors, are necessary for the amendment or suspension of these bylaws.
	
	A three-quarters majority affirmative vote of the Board of Directors and a two-thirds majority affirmative vote of all corporation members is necessary for the amendment of the bylaws. Certain aspects of the bylaws cannot be altered.
	
	\stepcounter{section}
	\stepcounter{section}
	
	\section{Prohibition Against Sharing Corporate Profits and Assets}
	This section reaffirms the non-profit nature of this corporation.
	
	\section{Dissolution}
	Three or more active member may propose to dissolve the Chapter by writing to the President, and a three-fourth majority affirmative vote of all active members is required to do so. The section then describes the process for returning unspent funds.
	
\end{document}